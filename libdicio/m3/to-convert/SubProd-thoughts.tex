\section{Basic definitions}

\newcommand{\sig}{\sigma}
\newcommand{\Sig}{\Sigma}

\newcommand{\rqq}{\mathbin{/\!/}}
\newcommand{\lqq}{\mathbin{\setminus\!\setminus}}

\subsection{Language quotients}

If $S$, $D$ are subsets of $\Sig^\ast$, then we define
the {\em right} and {\em left quotients of $S$ by $D$} as
\begin{eqnarray*}
  S \rqq D &=& 
    \rset{ x\in \Sig^\ast\suchthat (\forall y\in D\suchthat x y \in s) } \\
  D \lqq X &=& 
    \rset{ y \in \Sig^\ast\suchthat (\forall x\in D\suchthat x y \in s) }
\end{eqnarray*}

\subsection{Subproducts of a language}

If $S$ is a subset of $\Sig^\ast$, then a {\em subproduct} of $S$ is
a pair of languages $L,R\subseteq \Sig^\ast$ such that 
$L\cat R\subseteq S$.  

Note that the subproduct is the {\em pair} of languages, not their concatenation.

We say that a pair of languages $(L',R')$ is contained in another pair $(L,R)$
iff $L'\subseteq L$ and $R'\subseteq R$.  A {\em maximal subproduct}
of a language $S$ are the subproducts of $S$ that are maximal in this 
partial order.

\end{document}  
    
